\documentclass[conference]{IEEEtran}
\IEEEoverridecommandlockouts

% Packages
\usepackage{cite}
\usepackage{amsmath,amsfonts}
\usepackage{algorithm}
\usepackage{algpseudocode}
\usepackage{graphicx}
\usepackage{textcomp}
\usepackage{xcolor}
\usepackage{booktabs}
\usepackage{adjustbox}
\usepackage{listings}
\usepackage{verbatim}
\usepackage{multirow}

\def\BibTeX{{\rm B\kern-.05em{\sc i\kern-.025em b}\kern-.08em
    T\kern-.1667em\lower.7ex\hbox{E}\kern-.125emX}}

\begin{document}

    % Author info
    \title{Deliverable report \\
    \footnotesize \textit{Antonio Gelain: Mat: 258784, \\
    \texttt{antonio.gelain@studenti.unitn.it}, \\
    \texttt{antonio.gelain@eagletrt.it}, \\
    GitRepo: \texttt{https://github.com/Tonidotpy/GPU-Computing-2025-258784}}}

    \maketitle

    \begin{abstract}

    % GOAL
    % Improve performance of SpMV on GPU by leveraging \textbf{shared memory} and
    % other CUDA hardware-level features, and compare against your best
    % implementation from Deliverable 1 and cusparse implementation.

    \end{abstract}

    \begin{IEEEkeywords}
        Sparse Matrix, SpMV, CUDA, Parallelization, Storage Format
    \end{IEEEkeywords}

    \section{Introduction} 
     
    \section{Problem Statement}

    \section{Storage format}

    \section{Parallelization}

    \section{State of the Art}

    \section{Methodology and Contributions}\label{sec:methodology}

    \section{System Description and Experimental Set-up}
        \subsection{System Description}

        For the development of the project the following requirements were used
        inside the university cluster:
        \begin{itemize}
            \item SLURM 32.11.3
            \item CUDA 12.1
            \item GCC 12.3.0
            \item GNU Make 4.3
            \item OpenMP 17.0.6
            \item Eigen 12.3.0 (for the utilities)
            \item Python 3.9.18 (for the utilities) 
        \end{itemize}

        The project can be compiled using the provided Makefiles with GCC and
        CUDA compilers and can be run into the cluster via the SLURM workload
        manager utilities.

        \begin{table}[ht]
            \caption{System details}
            \label{tab:system_description}
            \centering
            \begin{adjustbox}{width=\columnwidth}
            \begin{tabular}{lllrl}
            \toprule
            \textbf{System} &  \textbf{Processor} & \textbf{Cores per Socket} & \textbf{RAM} & \textbf{Accelerator} \\
            \midrule
                laptop & Intel(R) Core(TM) i5-1035G1 CPU & 4 at 3.6 GHz & 8 GB & Intel Iris Plus Graphics G1 (Ice Lake) \\
                baldo & Intel(R) Xeon(R) Silver 4214 CPU & 12 at 2.2 GHz & 405 GB & \\
                edu-short & AMD EPYC 9334 32-Core Processor & 32 at 3.9 GHz & 810 GB & NVIDIA L40S \\
            \bottomrule
            \end{tabular}
            \end{adjustbox}
        \end{table}

        \subsection{Dataset description}
        
        Eight different matrices were selected as dataset, each with specific
        characteristics to evaluate the performance of the various
        implementations.

        The first five matrices are mostly symmetric and differ primarily in
        the number of non-zero elements, which increases progressively from one
        matrix to the next.

        The \texttt{heart1} matrix stands out for its \textbf{higher density},
        with non-zero values accounting to approximately 10\% of the total
        entries.

        The final two matrices are asymmetric and are designed to stress test
        different dimensions: one has a large number of rows, while the other
        has a large number of columns.

        \begin{table}[ht]
            \caption{Matrix Details}
            \label{tab:matrix_details}
            \centering
            \begin{adjustbox}{width=\columnwidth}
            \begin{tabular}{lrrrr}
            \toprule
            \textbf{Name} & \textbf{Rows} & \textbf{Columns} & \textbf{Non-Zeros} & \textbf{Symmetric} \\
            \midrule
                \textbf{1138\_bus} & 1,138 & 1,138 & 4,054 & Yes \\
                \textbf{bcsstk15} & 3,948 & 3,948 & 117,816 & Yes \\
                \textbf{bcsstk31} & 35,588 & 35,588 & 1,181,416 & Yes \\
                \textbf{pwtk} & 217,918 & 217,918 & 11,524,431 & Yes \\
                \textbf{cage15} & 5,154,859 & 5,154,859 & 99,199,551 & No \\
                \textbf{heart1} & 3,557 & 3,557 & 1,385,317 & No \\
                \textbf{relat8} & 345,688 & 12,347 & 1,334,038 & No \\
                \textbf{connectus} & 512 & 394,792 & 1,127,525 & No \\
            \bottomrule
            \end{tabular}
            \end{adjustbox}
        \end{table}

        \subsection{Experimental Set-up}

    \section{Experimental Results}

    \section{Conclusion}

    \bibliographystyle{IEEEtran}
    \bibliography{references}

\end{document}
