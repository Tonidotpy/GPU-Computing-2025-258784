\documentclass[conference]{IEEEtran}

\usepackage[margin=1in]{geometry} % full-width

% Author info
\title{Leveraging GPU Parallelism for Efficient CSR Sparse Matrix-Vector Multiplication}
\author{
    \IEEEauthorblockN{Antonio Gelain}
    \IEEEauthorblockA{
        \textit{Department of Information Engineering and Computer Science} \\
        \textit{University of Trento} \\
        Trento, Italy \\
        antonio.gelain@studenti.unitn.it
    }
}


\begin{document}
    \maketitle

    \begin{abstract}
        Sparse Matrix-Vector multiplication (SpMV) is a fundamental operation
        in fields such as scientific computing, graph analysis and machine
        learning, which can often becomes a performance bottleneck due to the
        large amount of data which needs to be processed.
        This paper will cover various implementations and optimization strategies
        used to achieve faster computation leveraging both the Graphics Processing
        Units (GPUs) and the Compressed Sparse Row (CSR) matrix format.

        % TODO: Review
        Although the CSR format is space efficient and widely used, its row-based
        structure presents difficulties for parallel execution on GPUs.
        Performance evaluations on a diverse set of sparse arrays drawn from
        real-world applications demonstrate significant speed gains over basic
        GPU implementations and competitive performance with more advanced libraries.
        The results highlight the feasibility of CSR-based SpMV for high-performance
        computing on GPUs, especially when combined with architecture-aware optimizations.
    \end{abstract}

    \section{Introduction} 
    
    \section{Problem Statement}


    \section{Conclusion}

\end{document}
